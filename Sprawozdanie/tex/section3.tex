\centering
{\LARGE\bfseries Wniosek \par}
\vspace{8mm}

\raggedright
Zwiększając długość wsztrzyknięcia, nie zmieniając liczby bakterii, otrzymujemy bardzej łagodne zakażenie, które się nadaje do szczepionki, a zmniejszając otrzymujemy niszczący wpływ na badane komórki. Także podbiór parametrów jest przydatny dla podtrzymywania kolonii bakterii oraz pełzaczy w badanym środowisku.
\vspace{2mm}

Podsumowując otrzymane wyniki, z pewnością możemy stwierdzić, że przy pomocy parametrów możemy dość realistycznie zamodelować rzeczywiste procesy, w tym \textit{zakażenie} organizmu oraz jego \textit{leczenie}. Wykonanie tego projektu pozwala zrozumieć jak wykorzystać język \textbf{Java} oraz \textit{narzędzie do automatyzacji} budowania projektów \textbf{Maven} dla realizacji zadań.
\newpage